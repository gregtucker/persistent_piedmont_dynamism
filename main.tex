\documentclass[12pt, oneside]{article}   	% use "amsart" instead of "article" for AMSLaTeX format
\usepackage{geometry}                		% See geometry.pdf to learn the layout options. There are lots.
\geometry{letterpaper}                   		% ... or a4paper or a5paper or ... 
\usepackage{graphicx}				% Use pdf, png, jpg, or eps§ with pdflatex; use eps in DVI mode
								% TeX will automatically convert eps --> pdf in pdflatex		
\usepackage{amssymb}
\usepackage{amsmath}
\usepackage{natbib}



\title{Persistent landscape dynamism at the fringes of decaying mountain ranges}
\author{Greg Tucker}
\date{}							% Activate to display a given date or no date

\begin{document}
\maketitle
%\section{}
%\subsection{}



\section*{NOTES}

Knepper 2005:

"Because the gravel is more
resistant than the underlying bedrock, modern streams are established over the Pleistocene drainage
divides, where the gravel was thinnest. Thicker gravel in the Pleistocene paleovalleys now caps modern
drainage divides, producing an inverted topography."

Fan has boulder, cobble, pebble gravel

RF 25 km2. E edge dissected by Rock, Walnut, Woman, Big Dry Cks, Barbara Gulch. N-trending K Laramie and Fox Hills pokes thru gravel in 4 places. He's looking @ aggregate potential of mtn front fans. 

Drilling, trenching, seismic, and ER showed big variation in thickness. DA Lindsey interps flash flood deposits; no ev DFs.

There's a small remnant "older gravel" 12-30 m above.



"a periglac~al climate fa\-oretl the generat~on
and rimsport of coarse clast~csediments along the early
Ple~stoccne Coal Creek Canyon and deposition where the
cteek ex~tedthe Front Range"

Max size fines down-fan. Peb-cobb-boulder. 70-77% Coal Ck qtzite.; granite & gneiss 12-21%; ss 2-8%; schist 2-4%; pegmatite & vein qtz.



\end{document}  
